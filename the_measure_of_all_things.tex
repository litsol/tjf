\environment style

\starttext

  \article {The Measure of All Things} {Thomas Fleming \| \currentdate}
    \noheaderandfooterlines % omit header and footer on this page

    Remember when a “Conservative” was someone opposed to change?
    “Any change, at any time, for any reason is to be deplored,” as
    The Duke of Cambridge (Victoria’s uncle) once declared.  Back in
    the 1950’s, the word got restricted to the meaning
    “anti-communist/capitalists who believed in a strong defense and a
    free economy, and it was embodied in the unlikely person of Barry
    Goldwater.  By the election of 1980, Conservatives had taken the
    initiative and were now the bold innovators in economic and
    foreign policy.  Most conservatives were delighted with the change
    of image—from Tory squire to progressive, from curmudgeon to
    optimist.

    I am not too sure.  Society needs curmudgeons, and somehow Ted
    Kennedy and Tip O’Neil do not fit the part.  A healthy skepticism
    is our best defense against the frivolous spirit of innovation
    that has infected since Tom Paine went up and down preaching
    blasphemy and the rights of man.  Many of the old Conservative—one
    is tempted to say Tory—positions, as ludicrous as they might seem
    today, were more deeply rooted in human nature and its traditions
    than supply side economics or global anti-Soviet commitments.  It
    would be hard, in these times, to drum much support for a crusade
    against women’s suffrage or to restore the rights of kings.
    However, another apparently ridiculous Conservative cause appears
    to have the support of the American people by a two to one margin:
    I mean opposition to the Metric System.

    In our preference for the old “inch-ounce” system, now known as
    the North American System, we Americans are in distinguished
    company with Barbados, Gambia, Oman, and South Yemen.  All the
    countries of Communist Eastern Europe and Socialist Western Europe
    have gone metric.  At home we are backing a trend support by our
    most enlightened constituencies: The National Education
    Association, the Association of Classroom Teachers, the Natio al
    Association of Secondary School Principals.  Between the NEA and
    South Yemen, it is admittedly a hard choice.

    With all this pressure from the Best and Brightest, our reluctance
    to going metric is hard to understand.  The fact is, our old
    English system was never legally established by Congress, although
    back in 1866 Andrew Johnson signed a law permitting the use of the
    Metric system, and in 1875 the U.S. was among the forty-eight
    signatories to the Convention du Mètre.  These dates should cause
    Southerners to take heart: the reconstruction of our system of
    measurement in America was first approved by those same ideologues
    who were busy reconstructing human nature in the South.

    Even so, it took Congress a hundred years to get around to passing
    a Metric Conversion Act in 1975.  This reluctance to interfere is
    uncharacteristic of Congress, especially since the Constitution
    expressly gives them the authority for “fixing the standard of
    weights and measures.”  President Washington, in is first message
    to Congress, requested them to make use of their mandate.  Thomas
    Jefferson actually did propose two separate plans: the first
    simply confirmed the old, English weights and measures; the second
    converted the traditional units to a decimal system.  It was
    probably not foreseen, even by Jefferson, that a much more radical
    reformation was about to be proposed and enforced by his friends
    in France.

    Enthusiasts for the Metric System claim to have exploded all the
    old conservative arguments; a superstitious belief that God
    somehow revealed the inch; an equally superstitious fear of
    anything that originated in the French Revolution; and the
    mistaken notion that conversion will cost too much.  Even people
    who are not much given to stroking rabbits’ feet or picking herbs
    that grow on a murderer’s grave by the light of a full moon, can
    still recognize that at the base of all superstitions is the
    acknowledgment that not everything is under man’s control.  In
    this sense the superstitious man shows a healthy and realistic
    respect for Natural and supernatural forces.  The apostles of
    Science, Progress, and the Metric System are a little like the
    advisors of King Canute, who assured him that authority alone
    could stop the tide.  If they cannot understand our superstitious
    reverence for the NAS (Notice how they systematize even the
    irrational with their acronyms!), we superstitious conservatives
    might find their rage to rationalize equally baffling.

    Now, no one is going to argue that God decreed the inch or defined
    the foot, but he did, we are reliably informed, make man among His
    latest and noblest creations.  We used to measure things—quite
    literally—on the human scale: in digits (the width of a finger),
    palms (four digits), spans (the distance between outstretched
    thumb and little finger, and cubits (from the tip of the middle
    finger to the elbow).  We still measure whiskey by fingers, though
    not (except in Texas) by palms, salt by pinches, and horses in
    hands.  In this sense, we may agree with Protagoras’ otherwise
    foolish solipsism that “man is the measure of all things.”

    We are more at home in a world measured by our hands and feet or
    by such homely measures as furlongs—the length of a furrow,
    originally, and acres—as much land as could be plowed by a team of
    oxen.  In a world increasingly dominated by machines and other
    impersonal forces, we can sympathize with the Cowboy Hall of Fame,
    which is suing the Federal Government for going metric, on the
    grounds that the West was won “inch by inch foot by foot, and mile
    by mile.”  Much of contemporary fashionable thought reveals a
    hatred of our species.  Baby seals, whales, and snail-darters are
    reverently (and rightly) defended by the same people who prate
    about a woman’s right to control her own body, even when it means
    the death of her child.  Man, we are told repeatedly, is the only
    evil thing in nature, the only creature that kills for pleasure.
    These urban misanthropes have never seen a weasel in a henhouse,
    as one of Melville’s characters remarks.

    This hatred of our species has an ancient smack to it.  It was the
    Docetists (among others) who rejected the Incarnation as
    crucifixion.  God was too “pure” to take on human flesh and
    suffering.  This mania for purity animated the Muslins in their
    destruction of icons and imagines, not to mention the English
    Puritans who vandalized churches and religious houses, closed the
    theaters, and desecrated the most potent political of man they
    possessed by chopping off the head of Charles I.  The friends of
    the Metric System are the living descendants of the Dervish, the
    Roundhead, and every other sworn enemy of the human race.  Under
    their prompting we now measure the world, not by the unworthy
    standard of the human body, but by 1/10,000,000th of the distance
    from equator to pole—or something fairly close to that figure.
    You see, the infallible savants made an uncharacteristically human
    error when they calculated the meter.  Later on, it seemed easier
    to retain the imperfect meter as it was than to go to the trouble
    of changing the whole system.

    Of course, not all traditional units of measurement were based on
    the human body.  Measures of volume evolved from the sizes of
    certain containers.  But even so, our old hodgepodge system, in so
    many respects typically Medieval, was a kind of living museum of
    Western history.  Our terms of measurement embodied the traditions
    of Greece, Rome, the Celts and Germans, even the Arabs.  What a
    sense of history a man can acquire from pondering the dram (Greek
    drachma), the mile (Latin mille passus—a thousand paces), the
    Saxon fathom, the French gill, and the Gaulish league.  What
    drives these reformers to hate their past so bitterly?  It is like
    the middle aged adolescent who still complains to his analyst
    about the emotional damage inflicted by his parents.  Such
    unreasoning hostility is always strange, but stranger still, when
    it is expressed in the name of reason.

    In fact, the Metric System originated in the fanatical worship of
    the goddess Reason.  It was Talleyrand himself who in 1790
    proposed the overhaul of the whole system of weights and measures.
    The unhappy Louis XVI was directed to order a scientific
    investigation and to invite his brother monarchs to send in their
    experts.  When George III expressed an understandable reluctance
    to collaborate with republicans, the French Academy of Science
    went on alone in the great work, setting up a commission which
    included the chemist Lavoisier.  The revolutionaries decided to
    honor the commission by disbanding it and Lavoisier by sending him
    to the guillotine.  In such a time, Lucan’s phrase about
    anarchy—Mensuraque juris vis erat (Force was the only measure of
    law)—might be aptly retranslated as: Measure was violence of law.

    We ought not to be embarrassed to find ourselves in company with
    an exemplary reactionary like the Sheik of Oman, and the complaint
    that our old-fashioned weights and measures isolate us in the
    world is simply a confession of weakness.  Back in 1821, John
    q. Adams may have had a point, when he suggested that we could not
    afford to jeopardize our trade with Britain by adopting the Metric
    System.  But in a world dominated by us and the Soviet Union,
    surely we can indulge our eccentricities.  The healthiest part of
    the American character has always been a willingness to persist in
    singularity, even when (as in the case of Prohibition) we are
    dreadfully wrong.  The fact is that simplification and
    standardization are part of the same science fiction mentality
    that would reduce the world’s peoples to one language—Esperanto or
    Newspeak—one government—the UN—or a benevolent Soviet Empire—and
    one culture—Public Television.

    The problem would be simpler if we could label the Metric System
    as Marxist.  Unfortunately the problem is much deeper.  We smell,
    inevitably, the whiff of brimstone the presence of Faust’s
    companion.  We have subjected everything else in our lives to
    rational numeration, why not measurement itself.  After all, this
    is the century that gave birth to serial music—a mathematical
    art-form utterly divorced from the requirements of the human ear.
    What are the Social Sciences, fundamentally, but an intrusion of
    numbers into places they do not belong?  The rearing of ouch
    children is now become a matter of preference tests, IQ tests, and
    achievement tests.  “Momentum” and opinion polls became the Djinn
    of political campaigns.  WE cannot escape this pernicious
    intellectualism even in the wasteland of televised sports, where
    the commentators solemnly intone statistics about RBIs, ERAs, and
    passing records, swamping the activity on the field in a downpour
    of numbers.  We even measure our lust in “vital statistics”—how
    big, how long, how often.  It is easy to imagine Mr. Hefner
    drooling over the rubberized abstractions of his 92-56-92
    (centimeters!) playmate.

    Going metric may well be as inevitable as socialism or genetic
    engineering.  Bayer aspirin is apparently 1 cm in diameter, and
    they already sell skis in metric sizes.  GM is expecting all
    passenger cars to be metric by 1982.  However, this is not simply
    a case of pragmatism versus tradition.  There was an estimated
    \$100 billion price tag attached, a figure which may turn out to be
    as accurate as the cost estimates in a defense contract.  Even
    though most Americans are not aware that the Metric Conversion Act
    of 1975 created a Metric Board to coordinate but not require the
    process of conversion, they are complaining bitterly about
    all-metric speed signs and 200 ml (6.8 oz) half pints of Jack
    Daniels.

    Complaints will not impress the reformers.  The loves of order and
    abstraction (One Meter, One World!) have too much at stake to
    consider backing down in the face of popular superstition.  Their
    universal vision is on the line—a tidy little world, sterilized
    against every taint of human history, with paradise just down the
    next kilometer.  On the other hand, they might listen to more
    active forms of persuasion.  We could build quite a bonfire with
    skis and Pontiacs, and we would have to do something with all
    those milliliters of Jack Daniels.

    \vfill

    This essay first appeared in The Southern Partisan, Fall 1981,
    which Clyde Wilson and I created and edited.  I have made a very
    few minor corrections and verbal alterations. The Measure of All
    Things

\stoptext

% finis
